\documentclass{llncs}
\usepackage{color}
\newcommand{\red}[1]{\textcolor{red}{#1}}
\newcommand{\blue}[1]{\textcolor{blue}{#1}}

\begin{document}


\section*{Preface}
This volume contains the proceedings of the 21st International
Conference on Relational and Algebraic Methods in Computer Science
(RAMiCS 2024), which was held at the Faculty of Arts of Charles University, 
Prague, Czech Republic, from 19 to 22 of August 2024. RAMiCS 2024 was organized by the Institute of Computer Science of the Czech Academy of Sciences, the Institute of Philosophy of the Czech Academy of Sciences, and the Faculty of Arts of Charles University.

The RAMiCS conferences series aims to bring together a community of
researchers to advance the development and dissemination of relation
algebras, Kleene algebras, and similar algebraic formalisms. Topics
covered range from mathematical foundations to applications as
conceptual and methodological tools in computer science and
beyond. More than 30 years after its foundation in 1991 in Warsaw,
Poland, RAMiCS, initially named "Relational Methods in Computer
Science", remains a main venue in this field. The series merged with
the workshops on Applications of Kleene Algebra in 2003 and adopted
its current name in 2009. Previous events were organized in Dagstuhl,
Germany (1994), Paraty, Brazil (1995), Hammamet, Tunisia (1997),
Warsaw, Poland (1998), Qu\'ebec, Canada (2000), Oisterwijk, The
Netherlands (2001), Malente, Germany (2003), St. Catharines, Canada
(2005), Manchester, UK (2006), Frauenw\"orth, Germany (2008), Doha,
Qatar (2009), Rotterdam, The Netherlands (2011), Cambridge, UK (2012),
Marienstatt, Germany (2014), Braga, Portugal (2015), Lyon, France
(2017), Groningen, The Netherlands (2018), Palaiseau, France (2020,
online), Marseille, France (2021), and Augsburg, Germany (2023).

RAMiCS 2024 attracted 21 submissions, of which 15 were
selected for presentation by the Program Committee. Each submission
was evaluated according to high academic standards by at least three
independent reviewers, and scrutinized further during two weeks of
intense electronic discussion. The organizers are very grateful to all
Program Committee members for this hard work, including the lively and
constructive debates, and to the external reviewers for their generous
help and expert judgments. Without this dedication we could not have
assembled such a high-quality program; we hope that all authors have
benefitted from these efforts.

Apart from the submitted articles, RAMiCS 2024 featured presentations from 
three invited speakers:
\begin{itemize}
\item Sergey Goncharov, Friedrich-Alexander University of Erlangen and Nürnberg, Germany;
\item Tomasz Kowalski, Jagiellonian University in Kraków, Poland;
\item Sarah Winter, IRIF, Université Paris Cité, France.
\end{itemize}
We are delighted that all three invited speakers accepted our invitation to 
present their work at the conference.

Last, but not least, we would like to thank the members of the RAMiCS
Steering Committee for their support and advice. The organizers of the conference are grateful to the following sponsors for their support: The Association for Logic, Language and Information; the Czech Science Foundation (projects 22-01137S, 22-23022L and 22-16111S); the Czech Society for Cybernetics and Informatics; the Institute of Computer Science of the Czech Academy of Sciences; the Institute of Philosophy of the Czech Academy of Sciences; and the Faculty of Arts of Charles University.
We also appreciate Ronan Nugent's help in publishing this volume with Springer. Finally, we
are indebted to all authors and participants for supporting this
conference.

 



~\bigskip


\noindent
\begin{minipage}[t]{.4\textwidth}
June 26, 2024\\
Prague
\end{minipage}%
\hfill
\begin{minipage}[t]{.4\textwidth}\flushright
  Uli Fahrenberg \\
  Wesley Fussner \\
  Roland Gl\"uck
\end{minipage}
\end{document}
